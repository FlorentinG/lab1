%part2a.tex
\subsection*{a. Stability of the solutions of an ODE-system of LCC-type}
The purpose of this section is to investigate the stability of an ODE while a parameter changes continuously.

The given third-order equation is the following : 
\begin{eqnarray*}
y'''+3y''+2y'+Ky =& 0 \\
y(0) =& 1\\
y'(0)=&1\\
y''(0)=&1
\end{eqnarray*}

It is possible to rewrite this as a system of ODE's, introducing new variables.
\begin{eqnarray*}
\textbf{u'}  = \left( \begin{array}{c}
y' \\ 
y'' \\ 
y'''
\end{array} \right) =& \left( \begin{array}{ccc}
0 & 1 & 0 \\ 
0 & 0 & 1 \\ 
-K & -2 & -3
\end{array}  \right) \left( \begin{array}{c}
y \\ 
y'\\ 
y''
\end{array} \right) = \textbf{Au} 
\end{eqnarray*}
The initial conditions are rewritten by :
\begin{eqnarray*}
\textbf{u(0)} =& \left( \begin{array}{c}
y(0) \\ 
y'(0)\\ 
y''(0)
\end{array} \right) = \left( \begin{array}{c}
1\\ 
1\\ 
1
\end{array} \right) 
\end{eqnarray*}

The solutions, computed analytically for different values of K, are given in figure \ref{result21}. It is possbile to see that for the first three values of K, the maximal amplitude of the blue curve (that is the function $y$) tends to decrease. This being an LCC system, we know that a perturbation will follow the same ODE so we can conclude that the system is stable for those values of K. On the other hand, for the last value ($K=8$), this amplitude increases.

\begin{figure}
\begin{center}
\includegraphics[scale=0.5]{result21.eps}
\caption{Solutions of the system in part 2a for various values of K}
\label{result21}
\end{center}
\end{figure}


%Ce truc insere mon code MATLAB (nice!) 
\lstinputlisting{LAB1_21.m}