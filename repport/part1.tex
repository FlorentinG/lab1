\section*{Part 1. Solution of ODE-systems with constant coefficients}

Starting from the described electrical circuit, we get the differential equation
$$L\ddot{q}+R\dot{q}+\frac{1}{C}q = E,\hspace*{5mm}q(0)=0,\hspace*{5mm}\dot{q}(0)=0,$$
after defining the variable $\dot{q}=i$. We easily rewrite this as a first order system of linear equations. Setting $$\textbf{y}= \begin{pmatrix}
y(1)\\
y(2)\\
\end{pmatrix} = \begin{pmatrix}
q\\
\dot{q}\\
\end{pmatrix},$$
this gives
%$$ \left.\begin{array}{lll}
%\dot{y}(1) & = & y(2)\\
%\dot{y}(2) & = & -\frac{R}{L}y(2) - \frac{1}{LC}y(1) + %\frac{E}{L}
%\end{array}\right. $$
$$\begin{pmatrix}
\dot{y}(1)\\
\dot{y}(2)\\
\end{pmatrix}= \begin{pmatrix}
0 & 1\\
- \frac{1}{LC} & -\frac{R}{L}\\
\end{pmatrix}\begin{pmatrix}
y(1)\\
y(2)\\
\end{pmatrix}
+ \begin{pmatrix}
0\\
E\\
\end{pmatrix}.$$
Lets define the matrix 
$$A=\begin{pmatrix}
0 & 1\\
- \frac{1}{LC} & -\frac{R}{L}\\
\end{pmatrix}$$ for some parameters $R$, $L$, $C$.

This problem is linear and the stability analysis of the problem is therefore quite straightforward. The problem will be stable if \red{(and only if?)} all the eigenvalues of A are negatives. Assume any combination of $R$, $L$, $C$ that are all positives, we have $\det(A)=\dfrac{1}{LC}>0$ and $\text{trace}(A)=-\frac{R}{L}<0$. This clearly ensures that all eigenvalues are negative and guaranties stability.

Below is our Matlab code for this question.
\lstinputlisting{LAB1_1.m}% for a file

The solution of the equation is shown on figure \red{todo} for al the values of R. We see that increasing R, tends to slow down the reaction of the system \red{check}. Obviously we observe that the derivative of the solution $q(t)$ is well represented on the graph.
 
